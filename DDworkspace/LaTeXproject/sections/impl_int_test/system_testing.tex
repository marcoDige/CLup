Once the system is implemented, unit tested, integrated and integration tested, it must be tested as a whole, to verify that all features have been developed correctly and that they comply with the functional and non-functional requirements defined in the RASD. In order to test the latter thing, it's necessary to do system testing. At this stage, the software should be as close to the final product as possible. This testing phase should not be performed exclusively by developers, being a black-box technique, it can therefore also include stakeholders and shareholders. We can split the system testing into the following stages:
\begin{itemize}
    \item \textit{Functional testing}: To make sure that functionality of the product is working as per the requirements defined, within the capabilities of the system
    \item \textit{Performance testing}: To identify bottlenecks affecting response time, utilization, and throughput. It needs an expected workload and an acceptable performance target before testing
    \item \textit{Load testing}: To expose bugs such as memory leaks, mismanagement of memory, or buffer overflow. It identifies also the upper limits of the components. In order to test the latter thing the load until threshold must be increased and the system with the maximum load it can operate for a long period, must be load
    \item \textit{Stress testing}: To make sure that the system recovers gracefully after failure. In order to test the latter thing, you have to try to break the system under test by overwhelming its resources or by taking resources away from it.
\end{itemize}