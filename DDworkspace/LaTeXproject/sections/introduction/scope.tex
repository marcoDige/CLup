The System aims to provide a solution to reduce overcrowding both inside and outside grocery stores.

\medskip
The application would work as a digital counterpart to the common situation where people who are in line for a service retrieve a number that gives their position in the queue.
The system should provide both the possibility to line up remotely (through an application on a mobile phone) and at the grocery store for those customers who do not have access to the required technology (\textbf{Line up functionality}).
Each customer that lined up should receive a number. Users should wait until his/her number is being called (or close to being called) to approach the store. This should reduce overcrowdings outside supermarkets.
Users can also scan a QR code when entering the grocery store, enabling the store manager to monitor entrances.

\medskip
In addition to lining up directly, an advanced function is offered. Customers can also book a visit to the supermarket, similarly to booking a slot for visiting a museum. The system should be able to schedule customer visits correctly given that each visit will last differently from the others.  
CLup can ask the customer details about his/her visit or it can compute an estimated duration from previous visits of the same user (\textbf{Book a visit functionality}).

\medskip
Ultimately, the system will have to be easy-to-use given that everyone needs to do grocery shopping and the more users will use the system remotely the more CLup will be effective.

\medskip
More detailed information can be found on the RASD.